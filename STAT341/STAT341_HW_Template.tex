\documentclass[]{article}
\usepackage{lmodern}
\usepackage{amssymb,amsmath}
\usepackage{ifxetex,ifluatex}
\usepackage{fixltx2e} % provides \textsubscript
\ifnum 0\ifxetex 1\fi\ifluatex 1\fi=0 % if pdftex
  \usepackage[T1]{fontenc}
  \usepackage[utf8]{inputenc}
\else % if luatex or xelatex
  \ifxetex
    \usepackage{mathspec}
  \else
    \usepackage{fontspec}
  \fi
  \defaultfontfeatures{Ligatures=TeX,Scale=MatchLowercase}
\fi
% use upquote if available, for straight quotes in verbatim environments
\IfFileExists{upquote.sty}{\usepackage{upquote}}{}
% use microtype if available
\IfFileExists{microtype.sty}{%
\usepackage{microtype}
\UseMicrotypeSet[protrusion]{basicmath} % disable protrusion for tt fonts
}{}
\usepackage[margin=1in]{geometry}
\usepackage{hyperref}
\hypersetup{unicode=true,
            pdftitle={Homework \#\# Template},
            pdfauthor={Your Name Here},
            pdfborder={0 0 0},
            breaklinks=true}
\urlstyle{same}  % don't use monospace font for urls
\usepackage{color}
\usepackage{fancyvrb}
\newcommand{\VerbBar}{|}
\newcommand{\VERB}{\Verb[commandchars=\\\{\}]}
\DefineVerbatimEnvironment{Highlighting}{Verbatim}{commandchars=\\\{\}}
% Add ',fontsize=\small' for more characters per line
\usepackage{framed}
\definecolor{shadecolor}{RGB}{248,248,248}
\newenvironment{Shaded}{\begin{snugshade}}{\end{snugshade}}
\newcommand{\KeywordTok}[1]{\textcolor[rgb]{0.13,0.29,0.53}{\textbf{#1}}}
\newcommand{\DataTypeTok}[1]{\textcolor[rgb]{0.13,0.29,0.53}{#1}}
\newcommand{\DecValTok}[1]{\textcolor[rgb]{0.00,0.00,0.81}{#1}}
\newcommand{\BaseNTok}[1]{\textcolor[rgb]{0.00,0.00,0.81}{#1}}
\newcommand{\FloatTok}[1]{\textcolor[rgb]{0.00,0.00,0.81}{#1}}
\newcommand{\ConstantTok}[1]{\textcolor[rgb]{0.00,0.00,0.00}{#1}}
\newcommand{\CharTok}[1]{\textcolor[rgb]{0.31,0.60,0.02}{#1}}
\newcommand{\SpecialCharTok}[1]{\textcolor[rgb]{0.00,0.00,0.00}{#1}}
\newcommand{\StringTok}[1]{\textcolor[rgb]{0.31,0.60,0.02}{#1}}
\newcommand{\VerbatimStringTok}[1]{\textcolor[rgb]{0.31,0.60,0.02}{#1}}
\newcommand{\SpecialStringTok}[1]{\textcolor[rgb]{0.31,0.60,0.02}{#1}}
\newcommand{\ImportTok}[1]{#1}
\newcommand{\CommentTok}[1]{\textcolor[rgb]{0.56,0.35,0.01}{\textit{#1}}}
\newcommand{\DocumentationTok}[1]{\textcolor[rgb]{0.56,0.35,0.01}{\textbf{\textit{#1}}}}
\newcommand{\AnnotationTok}[1]{\textcolor[rgb]{0.56,0.35,0.01}{\textbf{\textit{#1}}}}
\newcommand{\CommentVarTok}[1]{\textcolor[rgb]{0.56,0.35,0.01}{\textbf{\textit{#1}}}}
\newcommand{\OtherTok}[1]{\textcolor[rgb]{0.56,0.35,0.01}{#1}}
\newcommand{\FunctionTok}[1]{\textcolor[rgb]{0.00,0.00,0.00}{#1}}
\newcommand{\VariableTok}[1]{\textcolor[rgb]{0.00,0.00,0.00}{#1}}
\newcommand{\ControlFlowTok}[1]{\textcolor[rgb]{0.13,0.29,0.53}{\textbf{#1}}}
\newcommand{\OperatorTok}[1]{\textcolor[rgb]{0.81,0.36,0.00}{\textbf{#1}}}
\newcommand{\BuiltInTok}[1]{#1}
\newcommand{\ExtensionTok}[1]{#1}
\newcommand{\PreprocessorTok}[1]{\textcolor[rgb]{0.56,0.35,0.01}{\textit{#1}}}
\newcommand{\AttributeTok}[1]{\textcolor[rgb]{0.77,0.63,0.00}{#1}}
\newcommand{\RegionMarkerTok}[1]{#1}
\newcommand{\InformationTok}[1]{\textcolor[rgb]{0.56,0.35,0.01}{\textbf{\textit{#1}}}}
\newcommand{\WarningTok}[1]{\textcolor[rgb]{0.56,0.35,0.01}{\textbf{\textit{#1}}}}
\newcommand{\AlertTok}[1]{\textcolor[rgb]{0.94,0.16,0.16}{#1}}
\newcommand{\ErrorTok}[1]{\textcolor[rgb]{0.64,0.00,0.00}{\textbf{#1}}}
\newcommand{\NormalTok}[1]{#1}
\usepackage{graphicx,grffile}
\makeatletter
\def\maxwidth{\ifdim\Gin@nat@width>\linewidth\linewidth\else\Gin@nat@width\fi}
\def\maxheight{\ifdim\Gin@nat@height>\textheight\textheight\else\Gin@nat@height\fi}
\makeatother
% Scale images if necessary, so that they will not overflow the page
% margins by default, and it is still possible to overwrite the defaults
% using explicit options in \includegraphics[width, height, ...]{}
\setkeys{Gin}{width=\maxwidth,height=\maxheight,keepaspectratio}
\IfFileExists{parskip.sty}{%
\usepackage{parskip}
}{% else
\setlength{\parindent}{0pt}
\setlength{\parskip}{6pt plus 2pt minus 1pt}
}
\setlength{\emergencystretch}{3em}  % prevent overfull lines
\providecommand{\tightlist}{%
  \setlength{\itemsep}{0pt}\setlength{\parskip}{0pt}}
\setcounter{secnumdepth}{0}
% Redefines (sub)paragraphs to behave more like sections
\ifx\paragraph\undefined\else
\let\oldparagraph\paragraph
\renewcommand{\paragraph}[1]{\oldparagraph{#1}\mbox{}}
\fi
\ifx\subparagraph\undefined\else
\let\oldsubparagraph\subparagraph
\renewcommand{\subparagraph}[1]{\oldsubparagraph{#1}\mbox{}}
\fi

%%% Use protect on footnotes to avoid problems with footnotes in titles
\let\rmarkdownfootnote\footnote%
\def\footnote{\protect\rmarkdownfootnote}

%%% Change title format to be more compact
\usepackage{titling}

% Create subtitle command for use in maketitle
\newcommand{\subtitle}[1]{
  \posttitle{
    \begin{center}\large#1\end{center}
    }
}

\setlength{\droptitle}{-2em}

  \title{Homework \#\# Template}
    \pretitle{\vspace{\droptitle}\centering\huge}
  \posttitle{\par}
    \author{Your Name Here}
    \preauthor{\centering\large\emph}
  \postauthor{\par}
      \predate{\centering\large\emph}
  \postdate{\par}
    \date{Date Here}


\begin{document}
\maketitle

\section{About this template}\label{about-this-template}

This file provides a template that you should use for homework
assignments.

\subsection{R Markdown}\label{r-markdown}

Markdown is a simple formatting syntax for writing HTML, PDF, and Word
documents using a common interface. This template is setup to generate
pdf or MS Word (doc) files, since those are well-supported by the Canvas
system. For more details on using R Markdown see
\url{http://rmarkdown.rstudio.com}.

You can enter text into the document using the editor built into
RStudio. When you click the \textbf{Knit} button a document will be
generated that includes both content as well as the output of any
embedded R code chunks within the document. You can embed an R code
chunk like this:

\begin{Shaded}
\begin{Highlighting}[]
\CommentTok{# R code goes in blocks like this one.}
\CommentTok{# Use comments to help document your code}
\KeywordTok{head}\NormalTok{(iris)}
\end{Highlighting}
\end{Shaded}

\begin{verbatim}
##   Sepal.Length Sepal.Width Petal.Length Petal.Width Species
## 1          5.1         3.5          1.4         0.2  setosa
## 2          4.9         3.0          1.4         0.2  setosa
## 3          4.7         3.2          1.3         0.2  setosa
## 4          4.6         3.1          1.5         0.2  setosa
## 5          5.0         3.6          1.4         0.2  setosa
## 6          5.4         3.9          1.7         0.4  setosa
\end{verbatim}

In general, your homework responses should not include R code or raw R
output unless the question specifically requests them. To prevent them
from being printed, set \texttt{include=FALSE} in the code block header.
For example:

Documentation for other chunk options is available at
\url{https://yihui.name/knitr/options/}

\subsection{Headings}\label{headings}

If using this template for your homework, delete the text within these
sections and add your homework answers.

Type \texttt{\#\ Question\ (number\ here)} on a line to create a header
for each question. For problems with sub-parts, a secondary heading can
be created by typing \texttt{\#\#\ Part\ (letter\ here)} on a new line.

\subsection{Including Plots}\label{including-plots}

You can also embed plots, for example:

\includegraphics[width=0.75\linewidth]{STAT341_HW_Template_files/figure-latex/pressure-1}

Note that the \texttt{echo\ =\ FALSE} parameter was added to the code
chunk to prevent printing of the R code that generated the plot. It
would not work in this case to set \texttt{include=FALSE}, since that
would also hide the plot output.

\subsection{Including Math}\label{including-math}

RMarkdown allows you to easily incorporate formatted mathematical
notation using LaTeX. Mathematical expressions should be typed between
dollar signs (\$). For example, we can write \(x + y\).

\begin{itemize}
\tightlist
\item
  Subscripts can be added using underscore: \(x_1\)
\item
  Superscripts can be added using the caret character: \(x^2\)
\item
  Greek letters can be added by typing \textbackslash{} followed by the
  letter name: \(\beta\)
\item
  Equations can be centered and set off from text using two dollar
  signs: \[a^2 + b^2 = c^2\]
\end{itemize}

Note: Creating a MS Word document does not require an installation of
LaTeX on your computer. Creating a PDF output file does. A simple way to
do this to install `tinytex' by running the following commands:

\begin{Shaded}
\begin{Highlighting}[]
\CommentTok{# Only need to do this *once* for your computer}
\CommentTok{# in order to be able to create PDFs with math notation}
\KeywordTok{install.packages}\NormalTok{(}\StringTok{'tinytex'}\NormalTok{)}
\NormalTok{tinytex}\OperatorTok{::}\KeywordTok{install_tinytex}\NormalTok{()}
\end{Highlighting}
\end{Shaded}

\section{Question 1}\label{question-1}

\subsection{Part 1(a)}\label{part-1a}

Responses here

\subsection{Part 1(b)}\label{part-1b}

Responses here

\section{Question 2}\label{question-2}

Responses here.

\section{Question 3}\label{question-3}

The figure for this question is:

\includegraphics[width=0.75\linewidth]{STAT341_HW_Template_files/figure-latex/pressure2-1}

\section{Appendix}\label{appendix}

\begin{Shaded}
\begin{Highlighting}[]
\CommentTok{# Insert packages you need here}
\KeywordTok{library}\NormalTok{(knitr)}
\KeywordTok{library}\NormalTok{(ggplot2)}

\CommentTok{# R code goes in blocks like this one.}
\CommentTok{# Use comments to help document your code}
\KeywordTok{head}\NormalTok{(iris)}
\CommentTok{# Viewing the iris dataset, the second time}
\KeywordTok{head}\NormalTok{(iris)}
\CommentTok{# Code for the plot here}
\KeywordTok{ggplot}\NormalTok{(pressure) }\OperatorTok{+}\StringTok{ }
\StringTok{    }\KeywordTok{geom_point}\NormalTok{(}\KeywordTok{aes}\NormalTok{(}\DataTypeTok{x=}\NormalTok{temperature, }\DataTypeTok{y=}\NormalTok{pressure)) }\OperatorTok{+}
\StringTok{    }\KeywordTok{xlab}\NormalTok{(}\StringTok{"Temperature (deg C)"}\NormalTok{) }\OperatorTok{+}\StringTok{ }\KeywordTok{ylab}\NormalTok{(}\StringTok{"Pressure (mmHg)"}\NormalTok{)}
\CommentTok{# Only need to do this *once* for your computer}
\CommentTok{# in order to be able to create PDFs with math notation}
\KeywordTok{install.packages}\NormalTok{(}\StringTok{'tinytex'}\NormalTok{)}
\NormalTok{tinytex}\OperatorTok{::}\KeywordTok{install_tinytex}\NormalTok{()}
\CommentTok{# Code for the plot here}
\KeywordTok{ggplot}\NormalTok{(pressure) }\OperatorTok{+}\StringTok{ }
\StringTok{    }\KeywordTok{geom_point}\NormalTok{(}\KeywordTok{aes}\NormalTok{(}\DataTypeTok{x=}\NormalTok{temperature, }\DataTypeTok{y=}\NormalTok{pressure)) }\OperatorTok{+}
\StringTok{    }\KeywordTok{xlab}\NormalTok{(}\StringTok{"Temperature (deg C)"}\NormalTok{) }\OperatorTok{+}\StringTok{ }\KeywordTok{ylab}\NormalTok{(}\StringTok{"Pressure (mmHg)"}\NormalTok{)}
\end{Highlighting}
\end{Shaded}


\end{document}
