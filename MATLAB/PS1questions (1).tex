\documentclass[12pt]{amsbook}
%\usepackage[utf8]{inputenc}

\usepackage{tikz}
\newcommand*\circled[1]{\tikz[baseline=(char.base)]{
            \node[shape=circle,draw,inner sep=2pt] (char) {#1};}}
\usepackage{color}
\usepackage{amssymb}
\usepackage{amsfonts}
\usepackage{amsmath}
\usepackage{graphicx}
\usepackage{caption}
\usepackage{subcaption}

\usepackage[pdfpagelabels,hyperindex=false]{hyperref}


%%%%\usepackage{amsmidx}
\usepackage{appendix}
\usepackage{booktabs}
\usepackage[normalem]{ulem}
%%%%\makeindex{LABdriver6}
\usepackage[lastexercise]{exercise}

\newcommand{\xxj}{x^{(j)}}  

\newcommand{\xxi}{x^{(i)}}  
\newcommand{\bp}{\mathbb{P}}  
\newcommand{\br}{\mathbb{R}} 
 \newcommand{\brm}{\mathbb{R}^m} 
 \newcommand{\brn}{\mathbb{R}^n} 
 \newcommand{\bbrm}{$\mathbb{R}^m\,$} 
 \newcommand{\bbrn}{$\mathbb{R}^n\,$} 

\newcommand{\calr}{\mathcal{R}}  
\newcommand{\caln}{\mathcal{N}}  


\numberwithin{equation}{section}
% We'll use the equation counter for all our theorem environments, so
% that everything will be numbered in the same sequence.
%       Theorem environments
\theoremstyle{plain} %% This is the default, anyway
\newtheorem{thm}[equation]{Theorem}
\newtheorem{cor}[equation]{Corollary}
\newtheorem{lem}[equation]{Lemma}
\newtheorem{prop}[equation]{Proposition}
\theoremstyle{definition}
\newtheorem{defn}[equation]{Definition}
\newtheorem{Property}[equation]{Property}
\theoremstyle{remark}
\newtheorem{rem}[equation]{Remark}
\newtheorem{ex}[equation]{Example}
\newtheorem{notation}[equation]{Notation}
\newtheorem{terminology}[equation]{Terminology}

%\includeonly{LinearSystems,Exercises}
%\includeonly{Exercises}

%\includeonly{Eigenvectors,Exercises}
%\includeonly{Eigenvectors}
%\includeonly{PS8}

%\includeonly{GeometrySubspacestex}


%\usepackage{makeidx}
%\usepackage{robustindex}

\makeindex
 

\begin{document}

\section*{Exercises for Chapter 1}

\begin{Exercise}
%\begin{Exercise}[title={Computing an inverse}, difficulty = 0, label = rank1sum]
Let $A, B, C$ be given by
\begin{align*}
A =  \left[ 
\begin{array}{ccc} 
1& 1&  3\\ \noalign{\medskip}   
0& 1&  3\\ \noalign{\medskip}         
1& 2   & 2 
\end{array}
 \right],\,\,\, B =  \left[ 
\begin{array}{rrr} 
2& 0&  -1\\ \noalign{\medskip}   
0& 1&  3\\ \noalign{\medskip}         
0& 1   & -2 
\end{array}
 \right],\,\,\,
 C=  \left[ 
\begin{array}{rr} 
0& 1\\ \noalign{\medskip}   
1& 1\\ \noalign{\medskip}         
-1& -1   
\end{array}
 \right] ,\,\,\,
 D=  \left[ 
\begin{array}{rrr} 
0& 1 & 3\\ \noalign{\medskip}          
-1& -1   &-1
\end{array}
 \right] 
\end{align*}
%\end{Exercise}
For each problem find $X$ if possble.  If X doesn't exist say why.
\begin{itemize}
\item[a)]$X =  2A+3B$
\item[b)]$ X =7A + 2C$
\item[c)]$X =  BC$
\item[d)] $X = CB$
\item[e)]$ X = (A+2B)C$
\item[f)] Without actually computing the product, what is the size of the matrix $X = DC$
\end{itemize}

\end{Exercise}


\begin{Exercise}
Show that in general $AB \ne BA $ where  $A$ and $B$ are square.  It is sufficient to provide a single counter example for $A, B \in \mathbb{R}^{2 \times 2}$.
\end{Exercise}



\begin{Exercise}
%
%\begin{Exercise}[title={Computing an inverse}, difficulty = 0, label = rank1sum]
Let $A, b, x$ be given by
\begin{align*}
A =  \left[ 
\begin{array}{rr} 
1& 1\\ \noalign{\medskip}            
1& 2   
\end{array}
 \right],\,\,\, b =  \left[ 
\begin{array}{r} 
0\\ \noalign{\medskip}   
1 
\end{array}
 \right],\,\,\,
 x=  \left[ 
\begin{array}{r} 
x_1\\ \noalign{\medskip}   
x_2
\end{array}
 \right] 
\end{align*}
%\end{Exercise}
Write $$Ax=b$$ 
as a system of linear equations without matrix notation.
\end{Exercise}

\begin{Exercise}
%\begin{Exercise}[title={Computing an inverse}, difficulty = 0, label = rank1sum]
Let $A, b, x$ be given by
\begin{align*}
A =  \left[ 
\begin{array}{rrrr} 
1& 1& 0& 3\\ \noalign{\medskip}   
0& 1&  0&3\\ \noalign{\medskip}         
1& 2   &-1& 2 
\end{array}
 \right],\,\,\, b =  \left[ 
\begin{array}{r} 
2\\ \noalign{\medskip}   
-1\\ \noalign{\medskip}         
1 
\end{array}
 \right],\,\,\,
 x=  \left[ 
\begin{array}{r} 
x_1\\ \noalign{\medskip}   
x_2\\ \noalign{\medskip} 
x_3\\ \noalign{\medskip}                 
x_4   
\end{array}
 \right] 
\end{align*}
%\end{Exercise}
Write $$Ax=b$$ 
as a system of linear equations without matrix notation.
\end{Exercise}


\begin{Exercise}
Write the linear system
$$2x_1+x_2 = 7$$
$$3x_1-x_2 = 2$$
in the form $$Ax=b$$
What  are $A, b, x$?
\end{Exercise}

\begin{Exercise}
Write the linear system
$$5x_1-4x_2 = 1$$
$$3x_1+5x_2 = 2$$
$$6x_1-4x_2 = 8$$
$$7x_1+6x_2 = 2$$
$$2x_1-4x_2 = 9$$
$$8x_1+9x_2 = 2$$
in the form $$Ax=b$$
What are $A, b, x$?
\end{Exercise}

\begin{Exercise}
Let $A, B, C \in \mathbb{R}^{m \times n}$.
Provide a mathmatical proof of the associative law of addition for matrices, i.e., 
$$(A+B)+C = A+(B+C)$$  
Hint: you may use the fact that $(a+b)+c = a +(b+c) $ where $a, b, c$ are real
numbers.  Identify the $(i,j)$ component of the resulting matrix on each side of the equation
and show they are equal.
\end{Exercise}

\begin{Exercise}
The transpose of a matrix $A \in \mathbb{R}^{m\times n}$ (possibly not square) is the new matrix
$$X_{ij} = A_{ji}$$   The standard notation for this is $$X  = A^T$$  Given the matrices as defined in
Problem 1 find
\begin{itemize}
\item[a)] $A^T$
\item[b)] $B^T$
\item[c)] $C^T$
\item[d)] $D^T$
\end{itemize}
\end{Exercise}
\begin{Exercise}
The matrix $A$ is said to be symmetric if $A = A^T$.
\begin{itemize}
\item[a)] Construct an example of a symmetric matrix.
\item[b)] Can a matrix be symmetric if it is not square?
\item[c)] What is the plural of the word matrix?
\end{itemize}
\end{Exercise}

\begin{Exercise}
\begin{itemize}
\item[a)] Find the $(i,j)$ entry of the matrix
$AB$.
\item[b)] Find the $(i,j)$ entry of the matrix $(AB)^T$.
\item[c)] Find the $(i,j)$ entry of the matrix product $B^TA^T$.
\item[d)] Hence justify the equation
$(AB)^T = B^TA^T$
\item[e)] Is the matrix $XX^T$ symmetric?
\end{itemize}
\end{Exercise}


\bigskip

\begin{Exercise}[title={Scalar product}, difficulty = 0, label = exmm]
	Compute the scalar (or dot) product of the vectors
	$$u =  \left[ \begin {array}{c} 
	9 \\ \noalign{\medskip}       
	2 \\ \noalign{\medskip}        
	-1\end {array}
	\right], \quad v=
	\left[ \begin {array}{cccc} 
	0  \\ \noalign{\medskip}       
	2  \\ \noalign{\medskip}        
	1 \end {array}
	\right] 
	$$
\end{Exercise}

\bigskip

\begin{Exercise}[title={2-norm of a vector}, difficulty = 0, label = exmm]
\Question 	Let 
	$$ x =  \left[ \begin {array}{c} 
	3 \\ \noalign{\medskip}       
	4 \\ \noalign{\medskip}    
	    	-7 \\ \noalign{\medskip}        
	8\end {array}
	\right] 
	$$
	Find the 2-norm $\| x \|_2$.
	\Question 	Let 
	$$ x =  \left[ \begin {array}{c} 
	1 \\ \noalign{\medskip}       
	1 \\ \noalign{\medskip}           
	1\end {array}
	\right] 
	$$
	Find the 2-norm $\| x \|_2$.

	\Question 	Let 
	$$ x =  \left[ \begin {array}{c} 
	1/ \sqrt{3} \\ \noalign{\medskip}       
	1/ \sqrt{3}\\ \noalign{\medskip}           
1/ \sqrt{3}\end {array}
	\right] 
	$$
	Find the 2-norm $\| x \|_2$.

	
\end{Exercise}


\bigskip

\begin{Exercise}[title={Angle between vectors}, difficulty = 0, label = exmm]
\Question	Compute the angle between the vectors
	$$u =  \left[ \begin {array}{c} 
	1 \\ \noalign{\medskip}       
	2 \\ \noalign{\medskip} 
	8 \\ \noalign{\medskip}        
	-1\end {array}
	\right], \quad v=
	\left[ \begin {array}{cccc} 
	-1  \\ \noalign{\medskip} 
    2 \\ \noalign{\medskip}              
	-1  \\ \noalign{\medskip}        
	3 \end {array}
	\right] 
	$$
	providing your answer in both radians and degrees.
\Question
Compute the angle between the vectors
$$u =  \left[ \begin {array}{c} 
1 \\ \noalign{\medskip}       
2 \\ \noalign{\medskip}       
0\end {array}
\right], \quad v=
\left[ \begin {array}{cccc} 
0 \\ \noalign{\medskip}             
0  \\ \noalign{\medskip}        
3 \end {array}
\right] 
$$
providing your answer in both radians and degrees.

\Question
Compute the angle between the vectors
$$u =  \left[ \begin {array}{c} 
1 \\ \noalign{\medskip}       
2 \\ \noalign{\medskip}       
0\end {array}
\right], \quad v=
\left[ \begin {array}{cccc} 
-6 \\ \noalign{\medskip}             
-12  \\ \noalign{\medskip}        
0 \end {array}
\right] 
$$
providing your answer in both radians and degrees.


\end{Exercise}



\bigskip

\begin{Exercise}[title={A simple sum}, difficulty = 0, label = exmm]
Let
$$A =  \left[ \begin {array}{cccc} 
9  & 8  & 6 & 7 \\ \noalign{\medskip}       
2  & 2  & 7 & 8 \\ \noalign{\medskip}        
-1  & 3  & 1 & 1\end {array}
 \right] 
+
 \left[ \begin {array}{cccc} 
0  & 1  & 6 & 7 \\ \noalign{\medskip}       
2  & 2  & 2 & 8 \\ \noalign{\medskip}        
1  & -3  & 0 & 1\end {array}
 \right] 
$$
Find $A$.
\end{Exercise}



\begin{Exercise}[title={More addition}, difficulty = 0, label = exmm]
\Question 
Let
$$a =  \left[ \begin {array}{c} 
1   \\ \noalign{\medskip}       
2   \\ \noalign{\medskip}    
-2  \\ \noalign{\medskip}        
1   \end {array}
 \right] $$
and $$
b =  \left[ \begin {array}{c} 
0   \\ \noalign{\medskip} 
0    \\ \noalign{\medskip}        
5   \\ \noalign{\medskip}        
-1   \end {array}
 \right] 
$$
Find $a+b$.

\Question Let
$$u =  \left[ \begin {array}{ccccc} 
e  & 2  & -4 & 0 & -1  -1   \end {array}     
 \right] $$
and $$
v =  \left[ \begin {array}{ccccc} 
\pi  & 1  &1  &0 &0-1   \end {array}
 \right] 
$$
Find $u+v$.


\Question Let
$$A =  \left[ \begin {array}{ccc} 
\sqrt{2}  & 8  & 6  \\ \noalign{\medskip}       
2  & 1  & 7  \\ \noalign{\medskip}    
-12  & 1  & 1 \\ \noalign{\medskip}        
1  & 0  & -1 \end {array}
 \right] $$
and $$
B =  \left[ \begin {array}{cccc} 
0  & 1  & 2  \\ \noalign{\medskip} 
2  & -1  & 4  \\ \noalign{\medskip}        
-2  & 2  & 2  \\ \noalign{\medskip}        
1  & -1  & 1 \end {array}
 \right] 
$$
Find $A+B$.
\end{Exercise}


\bigskip

\begin{Exercise}[title={A simple product}, difficulty = 0, label = exmm]
\Question	Let
	$$M =  \left[ \begin {array}{ccc} 
	2  & 1  & 3 \\ \noalign{\medskip}            
	1  & -1  & 4 \end {array}
	\right], \quad N=
	\left[ \begin {array}{cc} 
	2  & 2 \\ \noalign{\medskip}             
	1  & -1\end {array}
	\right] 
	$$
\Question	Find the product $NM$.  Which of the products $MN$, $MM$ and $NN$ make sense?
\end{Exercise}



\begin{Exercise}[title={Matrix Multiplication}, difficulty = 0, label = exmm]
	Let
	$$A = \left( \begin{array}{rrrr}
	1 & -2 & 1 & 1\\
	2 & 2 & -3 &0\\
	1 & 2 & 1 &-1\end{array} \right)$$
	$$B = \left( \begin{array}{rrr}
	9 & 8 & 6 \\
	2 & 2 & 7\\
	1 & 3 & 1 \\
	-1 &2 &0\end{array} \right)$$
	Compute the product $AB$.
\end{Exercise}
\bigskip


\begin{Exercise}[title={Matrix Multiplication}, difficulty = 0, label = exmm]
	Let
	$$A = \left( \begin{array}{rrrr}
	-1 & 2 & 0 & -1\\
	3 & -4 & 2 &1\end{array} \right)$$
	$$B = \left( \begin{array}{rrr}
	-4 & 1 \\
	1 & 2\\
	1 & 1 \\
	1 &-7\end{array} \right)$$
	Compute the product $AB$.
\end{Exercise}
\bigskip


\bigskip

\begin{Exercise}[title={$AB \ne BA$}, difficulty = 0, label = exmm]
Let
$$A = \left( \begin{array}{rr}
1 & 0\\
2 & 2 \end{array} \right)$$
$$B = \left( \begin{array}{rr}
2 & 1\\
1 & 1 \end{array} \right)$$
\Question Find $AB$.
\Question Find $BA$.
\end{Exercise}
\bigskip

\bigskip

\begin{Exercise}[title={$AB \ne BA$}, difficulty = 0, label = exmm]
Let
$$A = \left( \begin{array}{rr}
-1 & 2\\
0 & 1 \end{array} \right)$$
$$B = \left( \begin{array}{rr}
9 & -6\\
4 & -2 \end{array} \right)$$
\Question Find $AB$.
\Question Find $BA$.
\end{Exercise}
\bigskip




\index{vector!position}
\begin{Exercise}[title={Position vectors}, difficulty = 0, label = exmm]
The representation of an $n$-tuple $x$ as a position vector involves drawing a line with
arrow fron the origin ${\mathbf{0}}$ to $x$.
Let
$$x = \left( \begin{array}{r}
1 \\
2 \end{array} \right), \quad
y = \left( \begin{array}{r}
-1 \\
3 \end{array} \right)
$$
\Question Draw the position vectors associated with the points $x$ and $y$.
\Question Compute $w = 3x-2y$ and plot this position vector.
\end{Exercise}
\bigskip


\index{vector!position}
\begin{Exercise}[title={Position vectors}, difficulty = 0, label = exmm]
Consider the vector starting at the point $(3,2)$ and ending at $(-2,1)$.
\Question Express this vector in terms of two position vectors, i.e., 
vectors that have the origin as an end point.
\Question Draw the three vectors.
\end{Exercise}

\bigskip
\begin{Exercise}[title={Transpose again}, difficulty = 0, label = 3T]
	Show that
	$$(ABC)^T = C^TB^TA^T$$
	What restrictions are there on the sizes of the matrices $A, B, C$?
	
	%Solution: 
	%$$A^T(A^{-1})^T = (A^{-1}A)^T = I$$
	%$$(A^{-1})^TA^T = (AA^{-1})^T = I$$
	
\end{Exercise}

\end{document}